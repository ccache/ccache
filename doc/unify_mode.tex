% !TeX encoding = UTF-8
% !TeX spellcheck = en_US
% !TeX program = lualatex


\documentclass{standalone}
\usepackage{comment}
\usepackage{minibox}

\usepackage{tikz}
\usetikzlibrary{calc}
\usetikzlibrary{decorations}
\usetikzlibrary{decorations.pathreplacing}
\usetikzlibrary{decorations.text}
\usetikzlibrary{decorations.pathmorphing}
\usetikzlibrary{decorations.markings}
\usetikzlibrary{fit}
\usetikzlibrary{arrows.meta}
\usetikzlibrary{shapes}
\usetikzlibrary{shadows.blur}
\usetikzlibrary{automata}
\usetikzlibrary{positioning}
\usetikzlibrary{petri}
\usetikzlibrary{chains}
\usetikzlibrary{fadings}
\usetikzlibrary{matrix}
\usetikzlibrary{patterns}
\usetikzlibrary{mindmap}
\usetikzlibrary{graphs}			% TikZ 3.0.0
\ifluatex
\usetikzlibrary{graphdrawing}	% TikZ 3.0.0
\fi
\usetikzlibrary{quotes}			% TikZ 3.0.0
\usetikzlibrary{babel}          % Vor polyglossia
\usetikzlibrary{hobby}
\usetikzlibrary{arrows.meta}
\usetikzlibrary{backgrounds}
\usetikzlibrary{bending}

\ifluatex
\usegdlibrary{trees}			% TikZ 3.0.0
\usegdlibrary{layered}			% TikZ 3.0.0
\usegdlibrary{force}			% TikZ 3.0.0
\usegdlibrary{circular}			% TikZ 3.0.0
\usegdlibrary{routing}			% TikZ 3.0.0
\fi

\pgfdeclarelayer{backbackbackground}
\pgfdeclarelayer{backbackground}
\pgfdeclarelayer{background}
\pgfdeclarelayer{foreground}
\pgfdeclarelayer{foreforeground}
\pgfsetlayers{backbackbackground,backbackground,background,main,foreground,foreforeground}
\tikzset{>={Stealth[round,flex,length=5pt 4.5 0.8]}} % Standard latex arrow tips are way too small
\tikzset{tight/.style={inner sep=0pt,outer sep=0pt,minimum size=0pt}}


\makeatletter
% pgf 'file' shape
\def\myfoldheight{0.5}
\def\myshapepath{
	\pgfextract@process\northwest{
		\southwest\pgf@xa=\pgf@x
		\northeast
		\pgf@x=\pgf@xa
	}

	\pgfextract@process\southeast{
		\southwest\pgf@ya=\pgf@y
		\northeast
		\pgf@y=\pgf@ya
	}

	\pgfextract@process\northfold{
		\pgfpointdiff{\southwest}{\northeast}
		\northeast
		\advance\pgf@x-\myfoldheight\pgf@y
	}

	\pgfextract@process\eastfold{
		\pgfpointdiff{\southwest}{\northeast}
		\northeast
		\advance\pgf@y-\myfoldheight\pgf@y
	}

	\pgfextract@process\fold{
		\northfold\pgf@xa=\pgf@x
		\eastfold
		\pgf@x=\pgf@xa
	}

	\pgfpathmoveto{\southwest}
	\pgfpathlineto{\northwest}
	\pgfpathlineto{\northfold}
	\pgfpathlineto{\eastfold}
	\pgfpathlineto{\southeast}
	\pgfpathclose
}

% compute an intersection point between a line and \myshapepath
% NOTE: Breaks inside \graph[layered layout]
\def\myshapeanchorborder#1#2{
	% #1 = point inside the shape
	% #2 = direction
	\pgftransformreset % without this, the intersection commands yield strange results
	\pgf@relevantforpicturesizefalse % don't include drawings in bounding box
	\pgfintersectionofpaths{
		\myshapepath
		%\pgfgetpath\temppath\pgfusepath{stroke}\pgfsetpath\temppath % draw path for debugging
	}{
		\pgfpathmoveto{
			\pgfpointadd{
				\pgfpointdiff{\southwest}{\northeast}\pgf@xc=\pgf@x \advance\pgf@xc by \pgf@y % calculate a distance that is guaranteed to be outside the shape
				\pgfpointscale{
					\pgf@xc
				}{
					\pgfpointnormalised{
						#2
					}
				}
			} {
				#1
			}
		}
		\pgfpathlineto{#1}
		%\pgfgetpath\temppath\pgfusepath{stroke}\pgfsetpath\temppath % draw path for debugging
	}
	\pgfpointintersectionsolution{1}
}
\def\myshapeanchorcenter{
	\pgfpointscale{.5}{\pgfpointadd{\southwest}{\northeast}}
}

% we could probably re-use some existing \dimen, but better be careful
\newdimen\myshapedimenx
\newdimen\myshapedimeny

\pgfdeclareshape{file}{
	% some stuff, we can inherit from the rectangle shape
	\inheritsavedanchors[from=rectangle]
	\inheritanchor[from=rectangle]{center}
	\inheritanchor[from=rectangle]{mid}
	\inheritanchor[from=rectangle]{base}

	% calculate these anchors so they lie on a coordinate line with .center
	\inheritanchor[from=rectangle]{west}
	\inheritanchor[from=rectangle]{east}
	\inheritanchor[from=rectangle]{north}
	\inheritanchor[from=rectangle]{south}

	% calculate these anchors so they lie on a line through .center and the corresponding anchor of the underlying rectangle
	\inheritanchor[from=rectangle]{south west}
	\inheritanchor[from=rectangle]{south east}
	\inheritanchor[from=rectangle]{north west}
	%    \inheritanchor[from=rectangle]{north east}

	% somewhat more special anchors. The coordinate calculations were taken from the rectangle node
	\inheritanchor[from=rectangle]{mid west}
	\inheritanchor[from=rectangle]{mid east}
	\inheritanchor[from=rectangle]{base west}
	\inheritanchor[from=rectangle]{base east}

	\backgroundpath{
		\myshapepath
	}

	\foregroundpath{
		\pgfpathmoveto{\northfold}
		\pgfpathlineto{\fold}
		\pgfpathlineto{\eastfold}
	}

	% This is from rectangle, i.e. without the fold.
	\anchorborder{%
		\pgf@xb=\pgf@x% xb/yb is target
		\pgf@yb=\pgf@y%
		\southwest%
		\pgf@xa=\pgf@x% xa/ya is se
		\pgf@ya=\pgf@y%
		\northeast%
		\advance\pgf@x by-\pgf@xa%
		\advance\pgf@y by-\pgf@ya%
		\pgf@xc=.5\pgf@x% x/y is half width/height
		\pgf@yc=.5\pgf@y%
		\advance\pgf@xa by\pgf@xc% xa/ya becomes center
		\advance\pgf@ya by\pgf@yc%
		\edef\pgf@marshal{%
			\noexpand\pgfpointborderrectangle
			{\noexpand\pgfqpoint{\the\pgf@xb}{\the\pgf@yb}}
			{\noexpand\pgfqpoint{\the\pgf@xc}{\the\pgf@yc}}%
		}%
		\pgf@process{\pgf@marshal}%
		\advance\pgf@x by\pgf@xa%
		\advance\pgf@y by\pgf@ya%
	}

	%    \anchorborder{
	%        \myshapedimenx=\pgf@x
	%        \myshapedimeny=\pgf@y
	%        \myshapeanchorborder{\myshapeanchorcenter}{\pgfpoint{\myshapedimenx}{\myshapedimeny}}
	%    }
}
\makeatother


\definecolor{citrine}{rgb}{0.89, 0.82, 0.04}


\begin{document}
\begin{tikzpicture}[font=\sffamily,row sep=4ex,column sep=2em]
\tikzset{file/.style={shape=file,draw,inner xsep=2ex}}
\tikzset{process/.style={draw,fill=lightgray,rectangle,align=center,text width=7em,blur shadow}}
\tikzset{decision/.style={draw,diamond,align=center,text width=4em}}
\tikzset{result/.style={font=\itshape}}
\tikzset{flow/.style={draw,->}}
\tikzset{run_cpp2/.style={flow,thick,red,->}}
\tikzset{run_nocpp2/.style={flow,thick,blue,->}}

\matrix[ampersand replacement=\&,column sep=3em,row sep=6ex] {
\node[file](src){source.c};                \& \\
\node[process](hash1){hash};               \& \\
\node[decision](lookup1){manifest lookup}; \& \node[process](directhash) {hash files from manifest file list}; \\
                                           \& \node[decision](directlookup) {cache lookup};                                    \& \node[result](directhit) {cache hit (direct)}; \\
\node[process](cpp){\texttt{cpp -P -MD}};  \& \\
\coordinate(cppoutput);                    \& \\
\node[process](hash2){hash};               \& \\
\node[decision](lookup2) {cache lookup};   \& \node[result](cpphit) {\minibox{cache hit\\(preprocessed)}}; \\
\node[process](cc) {\texttt{cc -c}};       \& \& \node[process](hashes) {dependent files and their hashes hashes to manifest}; \\
\node[file](o) {source.o};                 \& \& \node[process](to_cache) {to cache};                                          \& \node[result](miss) {cache miss}; \\
};
\matrix[ampersand replacement=\&,anchor=center] at (cppoutput.center) {
\node[file](i){source.i};	 \& \node[file](d1) {source.d};  \\
};


\path (src) edge[flow] (hash1);
\path (hash1) edge[flow] (lookup1);
\path (lookup1) edge[flow] node[midway,auto,sloped,font=\itshape] {miss} (cpp) ;
\path (cpp) edge[flow] (i);
\path (cpp) edge[flow] (d1);
\path (i) edge[flow] (hash2);
\path (d1) edge[flow] (hash2);
\path (hash2) edge[flow] (lookup2);
\path (lookup2) edge[flow] (cpphit);
\path (lookup2) edge[flow] node[midway,auto,sloped,font=\itshape] {miss} (cc);
\path (cc) edge[flow] (o);
\path (o) edge[flow] (to_cache);
\path[flow] (d1) -| (hashes);
\path (hashes) edge[flow] (to_cache);
\path (to_cache) edge[flow] (miss);
\path (hash2) -| ($(hash2)!0.5!(to_cache)$) |- (to_cache);

\path (lookup1) edge[flow] node[midway,auto,sloped,font=\itshape] {hit} (directhash);
\path (directhash) edge[flow] (directlookup);
\path (directlookup) edge[flow] (directhit);
\path[flow] (directlookup) |- node[pos=0.25,auto,sloped,font=\itshape] {miss} (cpp);


\path (i) edge[flow,bend right=30,run_nocpp2](cc);
\path (src) edge[flow,bend right=60,run_cpp2](cc);

\matrix[ampersand replacement=\&,draw,anchor=north east,column sep=0.3em,fill=white,row sep=0.6ex,nodes={anchor=west}] at ($(current bounding box.north east)+(-1cm,-1cm)$) {
  \path (0,0) edge[flow,run_cpp2] ++(1cm,0);   \& \node[font=\footnotesize]{\texttt{run\_second\_cpp=true}}; \\
  \path (0,0) edge[flow,run_nocpp2] ++(1cm,0); \& \node[font=\footnotesize]{\texttt{run\_second\_cpp=false}}; \\
};


\end{tikzpicture}
\end{document}
